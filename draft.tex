\RequirePackage{plautopatch}
\documentclass[uplatex,dvipdfmx,a4paper]{jsarticle}
\usepackage{ics-thesis}

% ===== Other packages =========================================================
\usepackage{otf}
\usepackage[hiresbb]{graphicx}
\usepackage{url}
\usepackage{multirow}
\usepackage{amssymb}
\usepackage{array}
\usepackage{arydshln}
\usepackage{multirow}
\usepackage{listings}
\usepackage[svgnames]{xcolor}
\usepackage{lscape}
\usepackage[caption=false,font=footnotesize]{subfig}
\usepackage{dblfloatfix}
\usepackage{ascmac}
\usepackage{xspace}  % for cite

% ==============================================================================

% ===== Package settings =========================================================
\definecolor{diffstart}{named}{Grey}
\definecolor{diffincl}{named}{Green}
\definecolor{diffrem}{named}{Red}

\lstdefinelanguage{diff}{
  basicstyle=\ttfamily\footnotesize\color{darkgray},
  frame=single,
  linewidth=1.00\textwidth,
  morecomment=[f][\color{diffstart}]{@@},
  morecomment=[f][\color{diffincl}]{+},
  morecomment=[f][\color{diffrem}]{-},
}
\lstset{
  language=diff,
  basicstyle=\ttfamily\footnotesize\color{darkgray},
  frame=single,
  linewidth=1.00\textwidth,
  breaklines=true,
  keywordstyle=\textbf,
  commentstyle=\color{green},
  keywordstyle=\color{blue},
  stringstyle=\color{red},
  tabsize=2,
  lineskip=-0.3zw
}

\hdashlinewidth=0.5pt
\hdashlinegap=1.0pt
% ==============================================================================


% リファレンス参照コマンド
\newcommand{\figref}[1]{図~\ref{#1}}
\newcommand{\tabref}[1]{表~\ref{#1}}
\newcommand{\chapref}[1]{\ref{#1}~章}
\newcommand{\secref}[1]{\ref{#1}~節}
\newcommand{\subsecref}[1]{\ref{#1}~項}

% ハイライトコマンド
\newcommand{\unmodified}[1]{{\textcolor{red}{#1}}}
\newcommand{\modified}[1]{{\textcolor{blue}{#1}}}
\newcommand{\needChange}[1]{{\textcolor{magenta}{#1}}}
\newcommand{\comment}[1]{{\textcolor{darkgray}{#1}}}
\newcommand{\TODO}[1]{{\textcolor{green}{#1}}}
% red,green,blue,cyan,magenta,yellow,black,gray,white,darkgray,lightgray,brown,lime,olive,orange,pink,purple,teal,violet

% footnote用のカウンタ
\newcounter{fcounter}
\setcounter{fcounter}{1}
\newcommand{\fcount}{\the\value{fcounter}\stepcounter{fcounter}}
\newcommand{\fcountin}{\the\value{fcounter}}

\let\oldcite\cite
\renewcommand{\cite}[1]{\xspace\oldcite{#1}}

% ===== Title page configuration ===============================================
\pagestyle{bachelorthesis}
\title{遺伝的アルゴリズムの時間的情報拡張による \\ 時間的効率の調査}
\author{皆森 祐希}
\supervisor{楠本 真二 教授}
\deadline{令和5年2月14日}
% ==============================================================================

\begin{document}

\titlepage

\pagestyle{empty}
\abstract {
  ソフトウェア開発における自動プログラム修正は開発工数の半分を占める作業といわれているデバッグの工数を削減することが期待されており,研究が盛んにおこなわれている.
  自動プログラム修正は,バグを含むプログラムに変更を加えることで用意したテストを通過するプログラムを出力する手法である.
  自動プログラム修正の手法として,遺伝的アルゴリズムに基づいて修正を行うものがある.
  ここで,自動プログラム修正における遺伝的アルゴリズムは,目的のプログラムが得られるまでプログラム文の挿入,削除,置換及び交叉を行う手法である.
  Macawは,遺伝的アルゴリズムを採用したAPRツールのひとつであるkGenProgの進化過程をツリー状に可視化するツールであり,遺伝アルゴリズムにおける膨大な変異プログラムの解析を容易にすることが期待される.しかし,kGenProg及びMacawには個体の生成時間を計測および表示する機能が実装されておらず,解となる経路が最適に生成されているかどうかを確かめることが困難である.
  そこで,kGenProg側に各個体を生成する時間を計測する処理を追加し,それをMacawで可視化することで全体の生成時間のうち解となるプログラムの生成経路にかかった時間を特定することにより,改善案を提供することが期待される.
}

\keyword {
  自動プログラム修正, 時間計測, JSON
}

% 目次
\clearpage
\pagestyle{plain}
\pagenumbering{roman}
\tableofcontents

% 図目次
\clearpage
\listoffigures

% 表目次
\clearpage
\listoftables

% 本文
\clearpage
\pagenumbering{arabic}

\newcommand{\apr}{APR}
\newcommand{\kgp}{kGenProg}
\newcommand{\mcw}{Macaw}
%%%%%%%%%%%%%%%%%%%%%%%%%%%%%%%%%%%%%%%%%%%%%%%%%%%%%%%%%%%%%%%%%%%%%%%%%%%%%%%%%%%%%%%%%%%%%%%%%%%
\clearpage
\section{はじめに}\label{sec:intro}
自動プログラム修正(\apr)は,人の手を介さずにソースコード中に含まれるバグを取り除く技術であり,研究が盛んである\cite{}

%%%%%%%%%%%%%%%%%%%%%%%%%%%%%%%%%%%%%%%%%%%%%%%%%%%%%%%%%%%%%%%%%%%%%%%%%%%%%%%%%%%%%%%%%%%%%%%%%%%

\clearpage
\section{研究背景}\label{sec:bg}
\clearpage
\section{準備}\label{sec:prep}
本章では,APR分野のブレイクスルーとなったGenProg\cite{l2011genprog}遺伝的アルゴリズム(GA)に基づく自動プログラム修正(APR) の生成過程について述べる.
APRは,テストケースと空プログラムを入力として,
GenProg\cite{le2011genprog}は,

\subsection{自動プログラム生成の流れ} \label{sec:prev_apg}

\subsection{既存のAPRツールの課題} \label{sec:prev_challenge}

既存のAPRツールの課題として,1回の修正実行に多くの時間がかかる点があげられる.
また,テストの実行時間を計測する機能は実装されているものの,1つの個体を生成するのにかかった時間を計測する機能が実装されていない.


%%%%%%%%%%%%%%%%%%%%%%%%%%%%%%%%%%%%%%%%%%%%%%%%%%%%%%%%%%%%%%%%%%%%%%%%%%%%%%%%%%%%%%%%%%%%%%%%%%%
\clearpage
\section{提案手法} \label{sec:prop}
\subsection{APRツールへの時間計測機能実装} \label{sec:impl}
提案手法では,\kgp のソースコード中内部のふるまいは変えずに,時間計測機能をはさみ,
\subsection{評価指標}\label{sec:index}
\subsubsection{FTR}\label{sec:FTR}
ビルドに失敗する個体が全体の生成時間に占める割合を求める指標として,{\bf FTR}(Failure Time Ratio, 失敗時間比率)を式\label{}で定義する.
FTR =  v
\subsection{}
具体的な例として,図\ref{fig:example}の生成木をもつプログラムの実行結果について考える.ここで,縦方向は世代を表しており,下に進むにつれてより後ろの世代である.また,横の列はある世代におけるすべての個体を表す.円及びバツ印はそれぞれビルドに成功した1つの個体,その世代でビルドに失敗したすべての個体を表す.右下の数字はその個体あるいは個体の集合の生成時間を表す.なおこの例におけるプログラムの総生成時間は800である.

\begin{figure}[t]
  \centering
  \includegraphics[width=\linewidth]{fig/astSample.pdf}
  \caption{生成結果の例}
  \label{fig:example}
\end{figure}

%%%%%%%%%%%%%%%%%%%%%%%%%%%%%%%%%%%%%%%%%%%%%%%%%%%%%%%%%%%%%%%%%%%%%%%%%%%%%%%%%%%%%%%%%%%%%%%%%%%
\clearpage
\section{実験} \label{sec:exp}
\subsection{概要}
本章では,提案手法を既存のAPRツールであるkGenProg\cite{higo2018kgenprog}を拡張して実装し,その効果を確認する.

%%%%%%%%%%%%%%%%%%%%%%%%%%%%%%%%%%%%%%%%%%%%%%%%%%%%%%%%%%%%%%%%%%%%%%%%%%%%%%%%%%%%%%%%%%%%%%%%%%%
\clearpage
\section{考察}



%%%%%%%%%%%%%%%%%%%%%%%%%%%%%%%%%%%%%%%%%%%%%%%%%%%%%%%%%%%%%%%%%%%%%%%%%%%%%%%%%%%%%%%%%%%%%%%%%%%
\clearpage
\section{妥当性の脅威}
\ref{sec:exp}章における実験において考えうる妥当性の脅威について論ずる.
今回調査対象としたプロジェクトは
プログラムの生成時間は
% また,kGenProg以外のツールを用いて実験を行った場合,


%%%%%%%%%%%%%%%%%%%%%%%%%%%%%%%%%%%%%%%%%%%%%%%%%%%%%%%%%%%%%%%%%%%%%%%%%%%%%%%%%%%%%%%%%%%%%%%%%%%
\clearpage
\section{おわりに}
本稿においては,自動プログラム修正における時間情報の拡張による

% =================================================================================================
% 謝辞
\clearpage
\acknowledgement

本研究を進めるにあたり,多くの方々からご支援およびご助言を賜りました.

楠本 真二 教授には,

肥後 芳樹 准教授には,議論を重ねに重ね,本研究の完成のご支援及び的確なご助言を賜りました.
心より感謝申し上げます.

柗本 真佑 助教には,テーマが決まらず途方に暮れていた際,鋭くも的確なご助言を賜りました.
深く感謝いたします.

本研究を進めるにあたり,古藤 寛大先輩には,賜りました.

本研究に至るまでに,講義,演習等でお世話になりました大阪大学基礎工学部情報科学科の諸先生方に,御礼申し上げます.

% =================================================================================================
% 参考文献
\clearpage
\bibliographystyle{junsrt}
\bibliography{references.bib}

\end{document}
